\documentclass{article}

\usepackage{xcolor}
\usepackage{amsmath}
\definecolor{redhighlight}{HTML}{FF0000}

\title{Fiche de Révision Algorithmique}
\author{Loureau Ryan}

\begin{document}
\maketitle 
\tableofcontents

\section{\textcolor{redhighlight}{Définitions}}
\begin{enumerate}
\item \textbf{Instance d'un Problème :}
Une Instance d'un problème est composée d'une valeur pour chaque donnée du problème.
\item \textbf{Algortihme :}
Un Algorithme est une méthode indiquant sans ambiguïté une suite finie d'actions mécaniques à éffectuer pour trouver la réponse à un problème
\item \textbf{Indécidabilité :}
Un problème pour lequel aucun algorithme n'existe est dit indécidable
\item \textbf{Complexité Algorithmique :}
La complexité algorithmique (ou coût ) est la mesure de l'éfficacité d'un algorithme, c'est a dire son temps d'éxecution et la place en mémoire utilisé. Cela permet de comparé deux algorithmes qui résolvent le même problèmes

\end{enumerate}

\section{\textcolor{redhighlight}{Problème Indécidable}}
\paragraph{Un problème indécidable veut dire qu'il n'existe pas d'algorithmes capable de reponde a n'importe quel instance du problème. Cela veut dire que pour tout algorithme, il existe une donnée du problème telle que soit l'algorithme ne s'ârrete pas ou la réponse est fausse.}

\section{\textcolor{redhighlight}{Problème Traitable}}
\paragraph{Un problème est dit traitable si il existe un algorithme pour le résoudre, il est considére non traitable lorsque l'algorithme met trop de temps a le résoudre ou utilise trop de mémoire. Mais on peut cherche un algorithmes qui donne une réponse approché ou qui ne reponde pas toujours.}

\section{\textcolor{redhighlight}{Notation de Landau}}
\begin{tabular}{|c|c|}
    \hline
    Notation & Signification \\
    \hline
    f = O(g) & f est bornée par g \\
    \hline
    f =   $\theta$(g) & f est du même ordre que g \\
    \hline
    f = o(g) & f est dominée par g \\

    \hline
\end{tabular}



\paragraph{Si f est bornée par g alors cela signifie que pour les grandes valeurs de n, f(n) ne dépasse pas k*g(n) }
\paragraph{Si f est du même ordre que g alors cela signifie que pour les grandes valeurs de n, f(n) est encadrée par k1 * g(n) et k2*g(n) ou dit autrement f = O(g) et g = O(f). }
\paragraph{Si f est dominée par g alors cela signifie que pour $\varepsilon$ aussi petit qu'on veut, et pour des grandes valeurs de n, f(n) ne dépasse pas $\varepsilon$ * g(n). Dit autrement pour des grandes valeurs de n alors f(n) est tout petit par rapport à g(n) }

\subsection{Utilisation pratiques}
\paragraph{Lorsqu'on aura une fonction f à étudier, on cherchera à trouver : 1 Une fonction simple g telle que  f = $\theta$(g) OU 2 à défaut, une fonction h telle que f = O(h).}
\paragraph{Lorsqu'on aura à comparer deux fonctions f et g, on cherchera à montrer : soit f = o(g), soit f= $\theta$(g)}

\subsection{Echelle de comparaison}
\begin{center} % Centrez le tableau sur la page
    \begin{tabular}{|c|c|}
        \hline
        Fonction & Nom \\
        \hline
        $O(1)$ & Constante \\
        \hline
        $O(\log n)$ & Logarithmique \\
        \hline
        $O((\log n)^n)$ & Polylogarithmique \\
        \hline
        $O(n)$ & Linéaire \\
        \hline
        $O(n \log n)$ & Log-linéaire \\
        \hline
        $O(n^2)$ & Quadratique \\
        \hline
        $O(n^3)$ & Cubique \\
        \hline
        $O(n^c)$ & Polynomiale \\
        \hline
        $O(n!)$ & Factorielle \\
        \hline
    \end{tabular}

\end{center}

\section{\textcolor{redhighlight}{Logarithme et exponentielle}}
\begin{center} % Centrez le tableau sur la page
\begin{tabular}{|c|c|}
\hline
\text{Fonction Exponentielle} & \text{Formule} \\
\hline
Exponentielle de base $e$ & $e^x$ \\
\hline
Exponentielle de base $a$ & $a^x$ \\
\hline
Logarithme naturel & $\ln(x)$ \\
\hline
Logarithme de base $a$ & $\log_a(x)$ \\
\hline
\text{Propriétés Exponentielles} & \\
\hline
Multiplication de bases égales & $a^x \cdot a^y = a^{x + y}$ \\
\hline
Division de bases égales & $\frac{a^x}{a^y} = a^{x - y}$ \\
\hline
Puissance d'une puissance & $(a^x)^y = a^{x \cdot y}$ \\
\hline
\text{Logarithmes} & \\
\hline
Changement de base & $\log_a(b) = \frac{\ln(b)}{\ln(a)}$ \\
\hline
Logarithme d'un produit & $\log_a(x \cdot y) = \log_a(x) + \log_a(y)$ \\
\hline
Logarithme d'un quotient & $\log_a\left(\frac{x}{y}\right) = \log_a(x) - \log_a(y)$ \\
\hline
Logarithme d'une puissance & $\log_a(x^y) = y \cdot \log_a(x)$ \\
\hline
Logarithme de 1 & $\log_a(1) = 0$ \\
\hline
\end{tabular}
\end{center}
\section{\textcolor{redhighlight}{Formules utiles}}
\paragraph{Somme des premiers entiers : $\sum_{i=1}^{n}$ i = $\frac{n(n+1)}{2}$ = O(n²)}
\paragraph{Somme des carrés des premiers entiers : $\sum_{i=1}^{n} i^2 = \frac{n(n+1)(2n+1)}{6}$ = O($n^3$)}
\paragraph{$n! =\sqrt{2n\pi}( \frac{n}{\exp})^n$}


\end{document}